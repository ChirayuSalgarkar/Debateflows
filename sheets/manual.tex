% Manual - How to Use This Flow Sheet

\thispagestyle{empty}

\begin{center}
    {\Huge\textbf{Flow Sheet Manual}}\\[0.3em]
    {\large How to use this NPDA debate flow document}
\end{center}

\vspace{1em}

\noindent\rule{\textwidth}{0.4pt}

\vspace{0.8em}

%% --- GETTING STARTED ---
\noindent{\Large\textbf{1. Setup}}
\vspace{0.3em}

\noindent Before each round, edit the metadata in \texttt{preamble.tex}:

\vspace{0.3em}
\begin{tabular}{|l|l|}
    \hline
    \rowcolor{gray90}
    \textbf{Command} & \textbf{What to change} \\
    \hline
    \verb|\tournament{...}| & Tournament name \\
    \verb|\roundnum{...}| & Round number (e.g.\ Round 1, Semis) \\
    \verb|\affteam{...}| & Aff team name / code \\
    \verb|\negteam{...}| & Neg team name / code \\
    \verb|\judge{...}| & Judge name \\
    \verb|\resolution{...}| & The resolution text \\
    \hline
\end{tabular}

\vspace{0.5em}
\noindent Fill in the \textbf{Cover Sheet} speaker names and case architecture during prep time.

\vspace{1em}

%% --- FLOW STRUCTURE ---
\noindent{\Large\textbf{2. Document Structure}}
\vspace{0.3em}

\noindent Each page is a separate flow sheet. should be able to fill columns, but \LaTeX is being finicky, so I'm deleting the how to use right now.

\vspace{0.3em}
% \begin{tabular}{|l|l|}
%     \hline
%     \rowcolor{gray90}
%     \textbf{Sheet} & \textbf{What to flow} \\
%     \hline
%     \textbf{Case Flow} & Aff case arguments (framework, contentions, solvency) \\
%     \textbf{DA Flow} & Disadvantages run by neg \\
%     \textbf{CP Flow} & Counterplans run by neg \\
%     \textbf{K Flow} & Kritiks run by neg \\
%     \textbf{Theory Flow} & Topicality, procedurals, theory arguments \\
%     \textbf{Overview Flow} & Big-picture framing, decision calculus, key voters \\
%     \hline
% \end{tabular}

\vspace{0.5em}
\noindent The six columns on each flow correspond to the six NPDA speeches:

\vspace{0.3em}
\begin{tabular}{|c|c|c|c|c|c|}
    \hline
    \cellcolor{affbg}\textcolor{affblue}{\textbf{PMC}} &
    \cellcolor{negbg}\textcolor{negred}{\textbf{LOC}} &
    \cellcolor{affbg}\textcolor{affblue}{\textbf{MG}} &
    \cellcolor{negbg}\textcolor{negred}{\textbf{MO}} &
    \cellcolor{negbg}\textcolor{negred}{\textbf{LOR}} &
    \cellcolor{affbg}\textcolor{affblue}{\textbf{PMR}} \\
    \hline
    7 min & 8 min & 8 min & 8 min & 4 min & 5 min \\
    \hline
\end{tabular}

\vspace{0.5em}
\noindent \textcolor{affblue}{Blue columns} = Aff speeches. \textcolor{negred}{Red columns} = Neg speeches. The column headers will automatically update with the speech names you set in the metadata.

\newpage

%% --- NOTATION GUIDE ---
\noindent{\Large\textbf{3. Notation Reference}}
\vspace{0.3em}

\renewcommand{\arraystretch}{1.4}
\begin{tabularx}{\textwidth}{|c|l|X|}
    \hline
    \rowcolor{gray90}
    \textbf{Symbol} & \textbf{Macro} & \textbf{Meaning} \\
    \hline
    \multicolumn{3}{|l|}{\cellcolor{gray90}\textit{Core Symbols}} \\
    \hline
    \ext & \verb|\ext| & \textbf{Extend} \\
    \drop & \verb|\drop| & \textbf{Dropped}\\
    \turn & \verb|\turn| & \textbf{Turn}\\
    \perm & \verb|\perm| & \textbf{Perm}\\
    \hline
    \multicolumn{3}{|l|}{\cellcolor{gray90}\textit{Turns \& Responses}} \\
    \hline
    \lturn & \verb|\lturn| & \textbf{Link turn}\\
    \iturn & \verb|\iturn| & \textbf{Impact turn}\\
    \nl & \verb|\nl| & \textbf{No link}\\
    \ow & \verb|\ow| & \textbf{Outweighs}\\
    \at & \verb|\at| & \textbf{Against / Answering} -- prefix for responses (e.g.\ \at\ C1) \\
    \hline
    \multicolumn{3}{|l|}{\cellcolor{gray90}\textit{Argument Labels}} \\
    \hline
    \link & \verb|\link| & \textbf{Link}\\
    \ix & \verb|\ix| & \textbf{Impact}\\
    \da & \verb|\da| & \textbf{Disadvantage}\\
    \cp & \verb|\cp| & \textbf{Counterplan}\\
    \kk & \verb|\kk| & \textbf{Kritik}\\
    \fw & \verb|\fw| & \textbf{Framework}\\
    \solv & \verb|\solv| & \textbf{Solvency}\\
    \hline
    \multicolumn{3}{|l|}{\cellcolor{gray90}\textit{Speed List Commands}} \\
    \hline
    \verb|\n| & \verb|\n| & \textbf{Next Level} (Level Down) — Opens a new nested list level (1. $\rightarrow$ a. $\rightarrow$ i.). \\
    \verb|\x| & \verb|\x| & \textbf{New Item} — Starts a new numbered or lettered argument line. \\
    \verb|\b| & \verb|\b| & \textbf{Back Out} (Level Up) — Closes the current list level and returns to the previous one. \\
    \hline
\end{tabularx}
\vspace{1em}


This is still a work in progress, Considering adding theory notation, also maybe just macros for perm do both perm do aff then do neg? 

\newpage

%% --- USAGE EXAMPLES ---
\noindent{\Large\textbf{4. Example Flow Entries}}
\vspace{0.3em}

\noindent Common patterns you'll write in cells:

\vspace{0.3em}
\begin{tabular}{|l|l|}
    \hline
    \rowcolor{gray90}
    \textbf{You write} & \textbf{Meaning} \\
    \hline
    \verb|\ext\ econ \ix\| & \ext\ econ \ix\ -- extend econ impact \\
    \verb|\drop\ \solv\| & \drop\ \solv\ -- dropped solvency \\
    \verb|\at\ C1: \nl| & \at\ C1: \nl\ -- answering contention 1 with ``no link'' \\
    \verb|\lturn\ -- helps aff| & \lturn\ -- helps aff -- link turn helps aff \\
    \verb|\ow\ -- scope + mag| & \ow\ -- scope + mag -- outweighs on scope and magnitude \\
    \verb|\perm\ do both| & \perm\ do both -- perm: do both the plan and alt \\
    \hline
\end{tabular}

\vspace{1em}

%% --- TIPS ---
\noindent{\Large\textbf{5. Tips}}
\vspace{0.3em}

\begin{itemize}
    \setlength{\itemsep}{0.2em}
    \item Probably use this with VSCode. You can open the folder as a workspace and easily switch between sheets. You can also use the LaTeX Workshop extension to compile and view PDFs directly in VSCode.
    \item To compile, just run \verb|pdflatex cover.tex| (or whichever sheet you're working on) in the terminal. Make sure you have \LaTeX\ installed. Maybe this works easier with Overleaf. I suspect yes, and then both people can see the flow at the same time. 
    \item Probably make the macros what you want.
    \item When switching sheets, maybe it's a good idea to quickly close pdf doc, open new panel, the press Ctrl-Alt-V or the icon with magnifying glass and split screen if using LaTeX Workshop in VSCode. Otherwise the pdf viewer might not update to the new sheet.
    \item Probably control-f is your friend, especially when you go to the actual tex files.
\end{itemize}

